% latex_assign_template.tex (Version 2021.2)
%
% A LaTeX template for assignments.
% Wouter Grouve (https://personen.utwente.nl/w.j.b.grouve)
% Faculty of Engineering Technology, University of Twente.
%
% Based on the excellent template from the Australian National
% Unversity, created by Timothy Kam in 2004. The original version can
% be found here: https://ctan.org/pkg/anufinalexam?lang=en
%
% Licence type: Free as defined in the GNU General Public Licence:
% http://www.gnu.org/licenses/gpl.html

\documentclass[a4paper,12pt,fleqn]{article}
\usepackage{lastpage}
\usepackage{xcolor}
\usepackage{amsmath}
\usepackage{fancyhdr}
\usepackage{enumitem}
\usepackage{graphicx}
\usepackage{tabularx}
\usepackage[skip=2pt,font=small]{caption}
\usepackage{environ}
\usepackage{mdframed}
\usepackage{hyperref}

% Booleans to show answers and to ask students to return the question form
\newif\ifshowanswers

% ==============================================================================
% Question command
%
\newcounter{question}
\newcommand*\question{%
  \stepcounter{question}%
  \paragraph{Question \thequestion}}

% ==============================================================================
% Answer boxes
%
\mdtheorem[outerlinewidth=2,roundcorner=10pt,
           leftmargin=0,rightmargin=0,
           backgroundcolor=yellow!40,outerlinecolor=blue!70!black,
           innertopmargin=\topskip,splittopskip=\topskip,
           ntheorem=true,]{answer_box}{Answer}[section]

\NewEnviron{answer}{
  \ifshowanswers
  \begin{answer_box*}
    \BODY
  \end{answer_box*}
  \fi}

% ==============================================================================
% Show answers?
%
\showanswerstrue
% \showanswersfalse


% ==============================================================================
%
% Course information
%
% ==============================================================================

\newcommand{\coursename}{Aircraft Structures}
\newcommand{\coursecode}{202000157}

\newcommand{\assigntype}{Homework Assignment \#1}
\newcommand{\duedate}{May 1st, 2021 at 17:00}
\newcommand{\teacher}{Wouter Grouve (\texttt{w.j.b.grouve@utwente.nl})}



% ==============================================================================
%
% Margins, header and footer
%
% ==============================================================================
\setlength{\topmargin}{0cm}
\setlength{\textheight}{9.25in}
\setlength{\oddsidemargin}{0.0in}
\setlength{\evensidemargin}{0.0in}
\setlength{\textwidth}{16cm}
\pagestyle{fancy}
\cfoot{\footnotesize{Page \thepage \ of \pageref{finalpage}
       -- \coursename \ (\coursecode)}}
\renewcommand{\headrulewidth}{0pt}
\renewcommand{\footrulewidth}{0pt}

\begin{document}

% ==============================================================================
%
% Header
%
% ==============================================================================

\noindent\makebox[\linewidth]{\rule{\textwidth}{0.4pt}}

\begin{center}
  \Large \textbf{\coursename} (\coursecode)
\end{center}

\begin{center}
  \large \assigntype{} \\
  \vspace{3mm}
\end{center}

\begin{center}
  Due date: \duedate\\
  Contact: \teacher
\end{center}

\noindent\makebox[\linewidth]{\rule{\textwidth}{0.4pt}}


% ==============================================================================
%
% Questions
%
% ==============================================================================

\question The two dimensional stress state at a particular point in a
structure equals \mbox{$\sigma_{\text{x}} =$ 350 MPa},
\mbox{$\sigma_{\text{y}} =$ 225 MPa} and \mbox{$\tau_{\text{xy}} =$
  100 MPa}.
\begin{enumerate}
\item{} Calculate the principal stresses and the orientation of the
  corresponding principal planes.
  \begin{answer}
    The principal stresses are: $\sigma_{\textrm{I}} = 405$ MPa and
    $\sigma_{\textrm{II}} = 170$ MPa, while the principal planes are
    oriented at $\theta = 0.51 \pm \pi/2$ or
    $\theta = 29^{\circ} \pm 90^{\circ}$.

    As a reminder, the equations for the principal stresses are:
    \begin{equation*}
      \sigma_{\textrm{I,II}} = \frac{\sigma_{\textrm{x}} + \sigma_{\textrm{y}}}{2} \pm
      \frac{1}{2}\sqrt{(\sigma_{\textrm{x}} - \sigma_{\textrm{y}})^2 + 4\tau^2_{\textrm{xy}}},
    \end{equation*}
    while the principal planes can be found using:
    \begin{equation*}
      \tan 2\theta = \frac{2\tau_{\textrm{xy}}}{\sigma_{\textrm{x}} - \sigma_{\textrm{y}}}.
    \end{equation*}
  \end{answer}
\item{} What is the magnitude of the shear stress acting on the
  principal planes?
  \begin{answer}
    The shear stress on the principal planes equals 0 MPa.
  \end{answer}
\end{enumerate}

\question The turbine blades in modern jet engines are subjected to
intense heat and extreme loads\ldots

\label{finalpage}
\end{document}